\chapter{Background} % Main chapter title

\label{Chapter2} % Change X to a consecutive number; for referencing this chapter elsewhere, use \ref{ChapterX}


\section{Specialization of Computers}
\todo{Should this paragraph be moved above?}
In the age of monolithic computers the size of entire rooms, it was expected of a computer to only be capable of serving a single function.
Before the technology could be miniaturized, every single computer had to be specialized for the job at hand.
As time passed and technology progressed, a single computer could be produced that served multiple functions as well as be adaptable programmable to handle new tasks that were not considered when the machine was originally built.
This gave rise to what was known as \enquote{Autonomic Computing}, a system of characteristics that are built into a computer to help it self-manage it's resources and adapt to it's administrator's requirements without the need to be redesigned for it's new purpose \todo{Correct citation format?}\citep{AutonomicComputing}.
Autonomic Computing was designed to combat the exponential complexity crisis that came from the widespread availability of computers in different disciplines, since now anyone who could afford a computer could program it for any purpose.
As new use cases were found for computers in day-to-day life and new technologies were being developed to interface with the world around us, the need for componentization became ever more apparent.
Researchers at IBM knew that the best way to address the looming problem of runaway complexity in computers was to develop a way for the computers to automatically interface with any new components that are installed, and to configure itself to only use what is necessary for the job at hand \citep[p.~43]{AutonomicComputing}.
This allowed hardware and software developers to focus on building their products to work with a common standard rather than having to manually integrate their products with every computer.

Once the idea of modularity began to take hold in the computer industry, attention was turned once again towards specializing individual computers.
Now that a computer's physical footprint can be minified, and peripheral components can be added and removed without a complete reengineering of the device, a single computer can be optimized to handle a single task without the overhead cost of building a monolithic computer \citep{Burbeck2007ComplexityAT}.
Now a computer can be specialized to serve a particular purpose or set of purposes while keeping development time comparatively short.
The world has seen this realized through numerous applications such as cell phones, designed to be user-friendly portable devices, computing clusters, built for high-performance computing, or even internet routers, which are built to be a plug-and-play solution for a problem posed to users of all levels of familiarity with computers.
Without this idea of Autonomic Computing, each of these devices would have to be reengineered from the ground up every time a new use case was developed.

This leads into the specialization of home computers in from the perspective of a consumer.
It used to be that a \enquote{home computer} was a device that was bought off the shelf as-is and served it's purpose.
Now a home computer can be a desktop PC that can be upgraded with time and seldom moves from it's place, or it could be a laptop that is portable and easy to cary around.
These specializations bring more choices for the consumer to pick a computer that best suits their needs, but they often come with compromises.

\subsection{Power of the Desktop}

The classical manifestation of a personal computer is the desktop computer.
Known for it's componentization, user repairability, and direct connections to power and the network, desktop computers are the best option for a user looking for a workhorse system.
Even though they aren't very portable, desktop computers allow consumers to access greater processor power for a lower cost compared to portable devices \citep{Meyer20145RS}.
Due to this, a desktop PC is often seen at one's home or place of work hardwired to the wall where it remains unmoving unless it needs upgrading, cleaning, or repairs.
Adding on the fact the desktop is modular, it can be easily customized to fit any user's needs, as well as enable particular parts to be replaced or upgraded without needing to rebuild the entire system.

While desktops were the staple of personal computers for decades, advancing technology and the rising need of portability and flexibility has led to the growth of laptops.


\subsection{Convenience of the Laptop}

As the world moved towards a more mobile lifestyle, the laptop shifted from being a luxury to being a necessity.


\todosection


\subsection{Rise of the Gaming Laptop}

\todosection


\section{Thin Clients}

\todosection


\section{Application to Modern Day}

\todosection