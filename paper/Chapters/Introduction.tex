\chapter{Introduction}

\label{Chapter1}

\section{Power and Portability}

As computers continue to grow to become more and more integrated into our daily lives, purchasing a computer has always been a balance.
Whether its deciding between something cheap or something long-lasting, a machine that is pre-built or a machine that is custom tailored and custom built, or, a tradeoff that has become much more common in recent years, something powerful or something portable.
As with anything, compromises can always be made and a middle ground may be found, but the only way to get the best of both worlds currently is to pay a price premium.
If someone were to look for a computer that is powerful, long-lasting, and upgradable, they would be directed towards a desktop computer.
However, if they needed something with the same power but portable, their only option would be to purchase a laptop costing much more than its desktop counterpart and often with limitations in upgradeability down the line.
Without purchasing both an expensive desktop and an expensive laptop, there currently is no way to get the benefits of both power and portability.


\section{Purpose of this Thesis}

The primary goal of this thesis is to create a portable hardware and software solution that can be used to control a desktop computer remotely while leveraging all of it's power.
There are existing software solutions that give a user remote control of their desktop computer from a different location, which will be introduced in Chapter \ref{Chapter3}, but such solutions may not provide a performant and high-quality  experience for the user.
These solutions will be tested while running highly demanding applications such as rendering animation, training Artificial Intelligence models, or developing video games to see if they are capable of working in such situations.

In other words, this thesis seeks to provide a user who already owns a powerful desktop computer with a mobile solution that gives them the ability to leverage it's power remotely.


\section{Thesis Overview}

This thesis is split into seven chapters.
Chapter \ref{Chapter2} introduces background knowledge surrounding existing computers, the shift to a more mobile computing world, and the modern day applications for this technology.
Chapter \ref{Chapter3} discusses the current the state of the art in terms of existing solutions for remote computer access, and presents a list of questions that this thesis seeks to answer.
Chapters \ref{Chapter4} and \ref{Chapter5} detail the requirements, design, and production of the hardware and software portions of the project respectively.
This design is tested against the requirements and the existing solutions in Chapter \ref{Chapter6}.
Finally, Chapter \ref{Chapter7} concludes the thesis with a summary of the project's innovations, limitations, and future research.